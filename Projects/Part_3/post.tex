
\documentclass{article}
\usepackage{fullpage}
\usepackage{url}
\usepackage{color}
\usepackage{listings}
\usepackage{longtable}

\renewcommand{\topfraction}{0.99}

\definecolor{dkgreen}{rgb}{0,0.6,0}
\definecolor{muave}{rgb}{0.58,0,0.82}
\definecolor{dkred}{rgb}{0.6,0,0}
\definecolor{dkblue}{rgb}{0,0,0.7}

\lstset{frame=none,
  language=C,
  % aboveskip=3mm,
  % belowskip=3mm,
  xleftmargin=2em,
  % xrightmargin=2em,
  showstringspaces=false,
  columns=flexible,
  basicstyle={\ttfamily},
  numbers=left,
  stepnumber=1,
  keywordstyle=\color{dkblue},
  directivestyle=\color{dkred},
  commentstyle=\color{dkgreen},
  stringstyle=\color{muave},
  breaklines=true,
  breakatwhitespace=true,
  tabsize=2
}

\lstdefinestyle{Output}{
  % aboveskip=3mm,
  % belowskip=3mm,
  xleftmargin=2em,
  % xrightmargin=2em,
  showstringspaces=false,
  columns=flexible,
  basicstyle={\ttfamily},
  numbers=none,
  keywordstyle=,
  directivestyle=,
  commentstyle=,
  stringstyle=,
  tabsize=2
}

\newcommand{\gradeline}{ \cline{1-2} \cline{4-4} ~\\[-1.5ex] }

\newenvironment{gradetable}{\begin{longtable}{@{}rrcp{5in}} \multicolumn{2}{l}{\bf Points} & & {\bf Description}\\ \gradeline}{\end{longtable}}

\newcommand{\mainitem}[2]{\pagebreak[2] {\bf #1} &&& {\bf #2}}
\newcommand{\mainpara}[1]{~ &&& {#1} }
\newcommand{\inneritem}[2]{~ & #1 && #2}
\newcommand{\innerpara}[1]{~ & ~ && #1}

\newcommand{\iseasier}{is$^\dagger$ }
\newcommand{\isextra}{is$^\ddagger$ }

\newcounter{rule}

\newcommand{\rulenumber}[1]{\refstepcounter{rule}R\therule\label{RULE:#1}}

\newcommand{\parser}{2}
\newcommand{\typecheck}{3}
\newcommand{\codegen}{4}


\title{COM S 440/540 Project part \typecheck}
\author{Type checking}
\date{}

\begin{document}

\maketitle


%======================================================================
\section{Requirements for part \typecheck}
%======================================================================

When executed with a mode of \typecheck,
your compiler should read the specified input file
and check that the file has correct C syntax
(as done in part~\parser),
and perform type checking on all expressions.
For any type errors (discussed in Section~\ref{SEC:types}),
write an error message to standard error, in the format
\begin{quote}
  \begin{tabbing}
		{\tt Type ch}\={\tt ecking error in file }\emph{filename}
		{\tt line }\emph{line number}
	\\
		\> \emph{Description}
  \end{tabbing}
\end{quote}
where the \emph{Description} should be as detailed as possible.
Your compiler should continue to catch and report errors
at the lexing and parsing stages
(i.e., errors from Parts 1 and \parser{} should remain).
After the first error,
your compiler may either make a ``best effort''
attempt to continue processing the input file, or exit.

If there are no errors,
then your compiler should create an output file
with extension {\tt .types},
that gives, for each statement of the form
\begin{quote}
\emph{expression} ;
\end{quote}
the type of the expression.
See Section~\ref{SEC:output}
for more details.

%======================================================================
\section{Type checking} \label{SEC:types}
%======================================================================


\subsection{Literals, identifiers, and functions}

Each expression will have a corresponding type.
For literals, the type may be inferred based on the literal:
character literals have type {\tt char},
integer literals have type {\tt int},
real literals have type {\tt float},
and string literals have type {\tt char[]}.
These may be modified by {\tt const}, see Section~\ref{SEC:constants}.
For global variables, local variables, and parameters,
the type should match how they are declared.
The return type of a function specifies the type of any call to that
function.
Regarding functions and identifiers,
an appropriate error message should be given for
\begin{itemize}
  \item
  declaring a variable, parameter, or struct member
    of type {\tt void};

  \item
  using a variable that has not been declared;

  \item
  declaring a local variable with the same name as another
  local variable or parameter of the same function;

  \item
  declaring a parameter with the same name as another parameter
  in the same prototype;

  \item
  declaring a global variable with the same name as another
  global variable;

  \item
  calling a function that has not been defined or declared as a prototype;

  \item
  calling a function with the incorrect number of parameters;

  \item
  calling a function with passed parameters whose types
  do not \emph{exactly} match the parameters given in its prototype
    (see Section~\ref{SEC:widening} to relax this slightly);

  \item
  a {\tt return} statement inside a function,
  that returns a value with a different type than the function's type
  (do not worry yet about \emph{missing} {\tt return} statements);

  \item
  giving more than one definition for the same function name.

\end{itemize}
Note that these type checking rules are more strict than standard C.
Extra features in Section~\ref{SEC:extras}
may adjust these rules or (more likely)
introduce additional rules,
to bring type checking closer to the C standard.


\subsection{Expressions}

\begin{table}[t]
\centering
\begin{tabular}{cccc|c}
  \# &
  \multicolumn{3}{c|}{ Operation } & Result type
\\ \hline
  \rulenumber{ite} &
  \multicolumn{3}{c|}{ $N$ ~ {\tt ?} ~ $T$ ~ {\tt :} ~ $T$ } & $T$
\\
  \multicolumn{4}{c}{~}
\\[2mm]
  \# &
  Left type & operator & Right type & Result type
\\ \hline
  \rulenumber{tilde}  &       & \verb|~| & $I$ & $I$ \\[1mm]
  \rulenumber{negate} &       & {\tt -} & $N$ & $N$ \\[1mm]
  \rulenumber{not}    &       & {\tt !} & $N$ & {\tt char}  \\[1mm]
  \rulenumber{makec}  &       & {\tt (char)} & $N$ & {\tt char} \\[1mm]
  \rulenumber{makei}  &       & {\tt (int)}  & $N$ & {\tt int}  \\[1mm]
  \rulenumber{makef}  &       & {\tt (float)} & $N$& {\tt float}
    \\[2mm]
  \rulenumber{mod} & $I$ &
    {\tt \%}, {\tt \&}, {\tt |}
    & $I$ & $I$
  \\[1mm]
  \rulenumber{plus} & $N$ &
    {\tt +}, {\tt -}, {\tt *}, {\tt /}
    & $N$ & $N$
  \\[1mm]
  \rulenumber{compare} & $N$ &
    {\tt ==}, {\tt !=}, {\tt >}, {\tt >=}, {\tt <}, {\tt <=},
    {\tt \&\&}, {\tt ||}
    & $N$ & {\tt char}
  \\[2mm]
  \rulenumber{preinc} &  &
    {\tt ++}, {\tt --}
    & $N$ & $N$
  \\[1mm]
  \rulenumber{postinc} & $N$ &
    {\tt ++}, {\tt --}
    & & $N$
  \\[1mm]
  \rulenumber{update} & $N$ &
    {\tt +=}, {\tt -=}, {\tt *=}, {\tt /=}
    & $N$ & $N$
  \\[1mm]
  \rulenumber{assign} & $T$ &
    {\tt =}
    & $T$ & $T$
  \\[2mm]
  \rulenumber{array} & $T${\tt[]} &
    {\tt [ int ]}
    & & $T$
\end{tabular}
\caption{Operations on types.
  $I \in \{\mathtt{char}, \mathtt{int}\}$ is any integer type,
  $N \in \{\mathtt{char}, \mathtt{int}, \mathtt{float}\}$ is any numeric type,
  and
  $T$ is any type excluding arrays and $\mathtt{void}$.
} \label{TAB:types}
\end{table}

All expressions should be checked for type errors.
Table~\ref{TAB:types} shows
how operators are defined for various types of operands,
and the resulting type.
Rules are numbered for later reference.
It also shows which explicit coercions, or casts, are allowed
(c.f. rules R\ref{RULE:makec}, R\ref{RULE:makei}, and R\ref{RULE:makef}).
Note that the variables $I$, $N$, and $T$ represent the same type
within a rule;
for example,
  rule~R\ref{RULE:plus} says the result of
  {\tt char + char} has type {\tt char}.
For a minimal implementation,
you may give an error if the operand types
do not \emph{exactly} match.
For example,
  rule~R\ref{RULE:plus} says the operands of {\tt +} must be
  the same type;
  thus an operation {\tt int + char} would produce a type error.
Rule~R\ref{RULE:array} is for array indexing,
and states both that
  it is an error to index an item that is not an array type,
  and that
  array indices must be integers (specifically, {\tt int}s).


\section{Extra features}
\label{SEC:extras}

\subsection{Prototype checking}
\label{SEC:protos}

Relax the type checking rules to allow function prototypes.
Note that this requires your parser to support function prototypes,
which was extra for part~\parser.

Specifically,
your compiler should allow any number of function prototypes
for a given function, as long as the return types,
number of parameters, and parameter types are consistent.
In other words, prototypes for the same function name
must match perfectly, except for parameter names.
A type checking error should be given otherwise.

In terms of type checking,
presence of a function prototype does not require that a
function definition also appears.
Relative to function prototypes,
the corresponding function definition may
appear before or after any of its prototypes,
or not appear at all.

\subsection{Automatic type widening}
\label{SEC:widening}

\begin{table}[t]
\centering
  \begin{tabular}{c|c}
    Original & Widened \\ \hline
    {\tt char} & {\tt int} \\
    {\tt int}  & {\tt float}
  \end{tabular}
\caption{Allowed type widenings.} \label{TAB:widenings}
\end{table}

Relax the type checking rules slightly by
automatically coercing types,
by widening only (according to Table~\ref{TAB:widenings})
as necessary.
Note that more than one widening may be possible, and desired.
For example, for operation {\tt char + float},
the left operand of type {\tt char} can be widened to {\tt int}
which can then be widened to {\tt float}.
Once the left operand has automatically been widened to type {\tt float},
rule~R\ref{RULE:plus}
  indicates that the resulting type is {\tt float}.

For this extra feature,
your compiler should automatically widen types any place it is needed:
for operations shown in Table~\ref{TAB:types},
for assignments,
for parameter passing,
and for function return values.
For example, if we have the following function defined:
\begin{quote}
  \begin{lstlisting}
    float widen(int a)
    {
      a += 'x';
      return a;
    }
  \end{lstlisting}
\end{quote}
then the expression in line~3 would automatically widen
the character constant \verb|'x'| to type {\tt int},
and the return statement in line~4 would automatically
widen variable {\tt a} from type {\tt int} to type {\tt float}
as required for the return type of function {\tt widen()}.
Calling the function as \verb|widen('q')|
would be allowed, as the character \verb|'q'| can
be automatically widened to type {\tt int} as required.


\subsection{Variable initialization}
\label{SEC:initialize}

Implement type checking for variable initialization.
Note that this requires your parser to support variable initialization,
which was extra for part~\parser.
Note that several variables may be declared and initialized at once:
\begin{quote}
  \begin{lstlisting}[numbers=none]
    void foo(int a)
    {
      int b = a,      /* OK */
          c,
          d = 0,      /* OK */
          e = d,      /* OK */
          f = 3.14,   /* ERROR, type mismatch */
          g;
    }
  \end{lstlisting}
\end{quote}

\subsection{Enforcing constants}
\label{SEC:constants}

Implement type checking for the {\tt const} modifier.
Note that this requires your parser to support the {\tt const}
modifier, which was extra for part~\parser.

For this extra feature,
an appropriate error message should be generated
for attempted modifications of any {\tt const} item.
This says the left side of any assignment operator
(rules R\ref{RULE:update} and R\ref{RULE:assign})
cannot be {\tt const},
and items with an increment or decrement operator
(rules R\ref{RULE:preinc} and R\ref{RULE:postinc})
cannot be {\tt const}.
The only exception is for variable initializations during declaration
(if this has been implemented):
\begin{quote}
  \begin{lstlisting}[numbers=none]
    void bar(int a)
    {
      const int b = a;  /* OK, declare and initialize  */
      const int c;
      c = 4;            /* ERROR, assignment to a const */
    }
  \end{lstlisting}
\end{quote}

Additionally,
when determining (and displaying) the type of an expression,
the following adjustments should be made.
\begin{itemize}
  \item
    The type of a string literal should be {\tt const char[]}.

  \item
    The type displayed for a variable should be {\tt const}
    if it was declared {\tt const}.

  \item
    The type displayed for an array element should be {\tt const}
    if the array is {\tt const}.

  \item
    All other operator rules (R\ref{RULE:ite} through R\ref{RULE:compare})
    should discard {\tt const} information.
    For example,
    it is legal to add a {\tt const int} to a {\tt const int} or to an {\tt int},
    and in both cases the resulting expression should have type {\tt int}.


\end{itemize}


\subsection{User-defined structs}
\label{SEC:userstructs}

Implement type checking for user-defined structs.
Note that this requires your parser to support user-defined structs,
which was extra for part~\parser.
This means that appropriate error messages should be given for the following.
\begin{itemize}
  \item
  It is an error to define two global structs with the same name.

  \item
  It is an error to define two local structs with the same name.

  \item
  When declaring a global variable, local variable,
  or parameter that is a struct type,
  it is an error if that struct type has not been defined.
\end{itemize}



\subsection{Member selection}
\label{SEC:members}

Implement type checking for member selection of structs.
Note that this requires your parser to support member selection
with the `` {\tt .} '' operator,
which was extra for part~\parser.
Practically, it also requires you to implement user-defined structs
as discussed in Section~\ref{SEC:userstructs},
otherwise this will be impossible to test.

This means to implement the following type checking rule
\begin{center}
  \begin{tabular}{cc|c}
    \# & ~ \qquad Operation \qquad ~ & Result type
  \\ \hline
    \rulenumber{members}
    & $S$ ~\verb|.|~ m & $T$
  \end{tabular}
\end{center}
where $T$ is the type of member ``m'' in struct $S$.
That is,
  for any expression of the form ``item.m'':
\begin{itemize}
  \item
  It is a type error if ``item'' is not a {\tt struct} type.

  \item
  It is a type error if the struct ``item''
  does not have a member named ``m''.

  \item
  Otherwise, ``item.m''
  has type as given by the type of member ``m'',
  as defined in the struct corresponding to ``item''.

  \item
  If a struct variable is {\tt const},
    then all of its members become {\tt const}
    (for students who implemented {\tt const}).

\end{itemize}
Note that this is complicated by the fact that ``item''
may also involve member selection and array indexing.
This extra feature is to type check expressions of the form
\begin{quote}
\begin{lstlisting}[numbers=none]
  window[3].upperleft.x = mouse.x.stack[mouse.x.top];
\end{lstlisting}
\end{quote}


%======================================================================
\section{Format of output} \label{SEC:output}
%======================================================================

For each statement of the form
\begin{quote}
  \emph{expression} {\tt ;}
\end{quote}
write a line of the form
\begin{quote}
  {\tt File} \emph{filename} {\tt Line } \emph{linenum}{\tt : expression has type}~\emph{type}
\end{quote}
to the output stream.
The \emph{filename} and \emph{linenum} given
should be the location of the expression statement
(specifically, the location of the ``\verb|;|'')
and \emph{type} should be the type of the expression,
as one of the four basic types
{\tt void}, {\tt char}, {\tt int}, or {\tt float},
or a user-defined {\tt struct structname}
(if that extra feature has been implemented),
followed by ``\verb|[]|'' if it is an array.
For example, the source lines from a file {\tt input.c}
\begin{quote}
\begin{lstlisting}[firstnumber=23]
    putchar(72);
    3.14159265 / 2.0;
    "Hello, world!";
\end{lstlisting}
\end{quote}
should generate the lines
\begin{quote}
  \begin{lstlisting}[style=Output]
  File input.c Line 23: expression has type int
  File input.c Line 24: expression has type float
  File input.c Line 25: expression has type char[]
  \end{lstlisting}
\end{quote}
in the output file {\tt input.types}.




%======================================================================
\section{Basic examples} \label{SEC:exmin}
%======================================================================

\subsection{Input: {\tt test1.c}}

\lstinputlisting{INPUTS/test1.c}

\subsection{Output: {\tt test1.types}}

\lstinputlisting[style=Output]{OUTPUTS_B/test1.types}

\subsection{Input: {\tt test2.c}}

\lstinputlisting{INPUTS/test2.c}

\subsection{Error for {\tt test2.c}}

\lstinputlisting[style=Output]{OUTPUTS_B/test2.error}

\subsection{Input: {\tt test3.c}}

\lstinputlisting{INPUTS/test3.c}

\subsection{Error for {\tt test3.c}}

\lstinputlisting[style=Output]{OUTPUTS_B/test3.error}


%======================================================================
\section{Extra feature examples} \label{SEC:exmax}
%======================================================================

\subsection{Input: {\tt extra3.c}}

\lstinputlisting{INPUTS/extra3.c}

\subsection{Error for {\tt extra3.c}}

\lstinputlisting[style=Output]{OUTPUTS_X/extra3.error}

\subsection{Input: {\tt extra4.c}}

\lstinputlisting{INPUTS/extra4.c}

\subsection{Output: {\tt extra4.types}}

\lstinputlisting[style=Output]{OUTPUTS_X/extra4.types}



%======================================================================
\section{Grading}
%======================================================================

Our testing script, {\tt TypesTest.sh}, is published on the course
git repository, along with \LaTeX{} source for these specifications.
Students are \emph{strongly} encouraged to test their code
using the script, especially on {\tt pyrite.cs.iastate.edu},
before submission.
The script works in the same way as the previous grading scripts,
{\tt LexTest.sh} and {\tt ParseTest.sh}.


\noindent
\begin{gradetable}
  \mainitem{15}{Documentation}
  \\[1mm]
  \inneritem{3}{\tt README.txt}
  \\[1mm]
  \innerpara{%
    How to build the compiler and documentation.
    Updated to show which part \typecheck{} features are implemented.
  }
  \\[1mm]
  \inneritem{12}{\tt developers.pdf}
  \\[1mm]
  \innerpara{%
    New section for part \typecheck, that explains
    the purpose of each source file,
    the main data structures used (or how they were updated),
    and gives a high-level overview of how the various
    features are implemented.
  }
  \\[4mm]

  \mainitem{6}{Ease of building}
  \\[1mm]
  \innerpara{%
    How easy was it for the graders to build your compiler and
    documentation?
    For full credit, simply running ``{\tt make}''
    should build both the documentation and the compiler executable,
    and running ``{\tt make clean}'' should remove
    all generated files.
  }
  \\[4mm]

  \mainitem{8}{Still works in modes 0, 1, and \parser}
  \\[1mm]
  \mainpara{%
    We will test earlier modes only using source files
    that are syntactically correct,
    with no type errors.
  }
  \\[4mm]

  \mainitem{66}{Type checking}
  \\[1mm]
  \inneritem{4}{Literals}
  \\[1mm]
  \inneritem{10}{Identifiers (global variables, local variables, parameters)}
  \\[1mm]
  \inneritem{10}{Function calls}
  \\[1mm]
  \inneritem{4}{Function returns}
  \\[1mm]
  \inneritem{4}{Unary operators}
  \\[1mm]
  \inneritem{4}{Casts}
  \\[1mm]
  \inneritem{6}{Binary operators: arithmetic}
  \\[1mm]
  \inneritem{6}{Binary operators: comparison and logic}
  \\[1mm]
  \inneritem{6}{Assignment and update operators}
  \\[1mm]
  \inneritem{4}{Increment and decrement}
  \\[1mm]
  \inneritem{4}{Array indexing}
  \\[1mm]
  \inneritem{4}{Ternary operator}
  \\[4mm]

  \mainitem{35}{Extra features}
  \\[1mm]
  \inneritem{5}{Prototype checking (c.f. Section~\ref{SEC:protos})}
  \\[1mm]
  \inneritem{5}{Automatic widening (c.f. Section~\ref{SEC:widening})}
  \\[1mm]
  \inneritem{5}{Variable initializations (c.f. Section~\ref{SEC:initialize})}
  \\[1mm]
  \inneritem{5}{Constants (c.f. Section~\ref{SEC:constants})}
  \\[1mm]
  \inneritem{5}{User-defined structs (c.f. Section~\ref{SEC:userstructs})}
  \\[1mm]
  \inneritem{5}{Struct member selection (c.f. Section~\ref{SEC:members})}
  \\[1mm]
  \inneritem{5}{{\tt const} with {\tt struct}}
  \\[4mm]

  \gradeline
  \mainitem{100}{Total for students in 440 (max points is 120)}
  \\
  \mainitem{115}{Total for students in 540}
\end{gradetable}


\section{Submission}

\begin{table}[h]
\centering

  \begin{tabular}{lcr}
    {\bf Part} & ~~~~ & {\bf Penalty applied} \\ \cline{1-1}\cline{3-3}
    Part 0 && 30\% off \\
    Part 1 && 20\% off \\
    Part \parser && 10\% off
  \end{tabular}

\caption{Penalty applied when re-grading}
\label{TAB:penalties}
\end{table}

Be sure to commit your source code and documentation to your
git repository, and to upload (push) those commits to the server
so that we may grade them.
In Canvas,
indicate which parts you would like us to re-grade for reduced credit
(see Table~\ref{TAB:penalties} for penalty information).
Otherwise, we will grade only part \typecheck.




\end{document}

