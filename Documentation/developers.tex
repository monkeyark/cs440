\documentclass{article}
\usepackage[utf8]{inputenc}

\title{cs440}
\author{Zhi Wang}
\date{January 2023}

\begin{document}

\maketitle

\section{Introduction}
Next Generation Compiler But God Know If It Would Work.
NGCBDKIIWW is a compiler for a subset of the C programming language.

\section{Compile}
Move to the Source directory.
All files will be generated in Source directory after compile.
In the Source, use\\
{\tt make} to compile the executable\\
{\tt make clean} to remove compiled binary and linking files\\
{\tt make clobber} to remove all generated files

\section{Useage}
The part mark with * is under development.

\begin{verbatim}
	Usage:
		mycc -mode [options] infile

	Valid modes:
		-0: Version information
		-1: C Lexer
		*-2: C parser
		*-3: Type checking
		*-4: Code generation: Expressions
		*-5: Part 5

	Valid options:
		-o outfile: write to outfile instead of standard output
\end{verbatim}


\section{Part 0 - Version information}
{\tt mycc.cpp}:
Main function of the program with takes arguments from standard input 
as switch to enter specified mode.
Version information of the project.

\section{Part 1 - C Lexer}
{\tt lex.h}:
Header file of the lexer. 
Defined token attributes and error messages.
{\smallskip}
{\tt lex.cpp}: 
A lexer that read the current and 
next char from the input file and then output token
when reach a token terminator or end of file.

\begin{verbatim}
	void lex_file(string path)
	void lex_text(string text, string fname)
\end{verbatim}
take input from file and process the input character by character from {\tt fname}.
The function use multiple flag variables to check if the 
current char is in comment of c/c++ or in literals of string/character.
There is also flag to check if the current char is in a real literals or include path.
When a flag is enable, the current char will be consumed and until 
reach the end of the token by terminate checking.
Whenever a token is found, it will calls
\begin{verbatim}
	void output_token(string lexeme, int line, string fname)
\end{verbatim}
to search if the current token is vaild one and output the token with its tokenid.






\section{Part 2 - C Parser}

\section{Part 3 - Type checking}

\section{Part 4 - Code generation: Expressions}

\section{Part 5 - }


\end{document}
