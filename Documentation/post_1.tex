\documentclass[10pt]{article}
\usepackage{fullpage}
\usepackage{url}
\usepackage{color}
\usepackage{listings}

\definecolor{dkgreen}{rgb}{0,0.6,0}
\definecolor{dkred}{rgb}{0.6,0,0}
\definecolor{dkblue}{rgb}{0,0,0.7}

\usepackage{tikz}
\usetikzlibrary{automata, positioning, arrows}
\tikzset{%
  node distance=2.5cm,  
  initial text={},      
  every state/.style={  
    semithick},
  double distance=2pt,  % Accept state appearance
  every edge/.style={   
    draw,
    ->,
    >=stealth',
    auto,
    semithick} %
}

\lstdefinestyle{jvm}{
  % aboveskip=3mm,
  % belowskip=3mm,
  xleftmargin=2em,
  % xrightmargin=2em,
  showstringspaces=false,
  columns=flexible,
  basicstyle={\ttfamily},
  numbers=left,
  moredelim=[s][\color{black}]{Ljava}{;},
  morecomment=[l][\color{dkgreen}]{;},
  morecomment=[l][\color{magenta}]{;;},
  keywords={class,public,static,super,method,code,end},
  keywordstyle=\color{dkblue},
  breaklines=true,
  breakatwhitespace=true,
  tabsize=8
}

\renewcommand{\thepage}{~}

\title{COM S 440/540 Homework 1}
\date{}
\author{Lexical Analysis 1}

\begin{document}

\maketitle

\noindent
Reminder: present your own work and properly cite any sources used.
Solutions should be written satisfying the \emph{other student viewpoint},
and must be prepared using \LaTeX.
\renewcommand{\thepage}{~}
 
%============================================================
\section*{Question~1~\hfill 25--30 points}
%============================================================

Implement a C function,
with appropriate comments to explain how your code works,
to read an unsigned integer with commas as the thousands separator.
The function should read from a file,
and write the digits (without the commas) in order into a character buffer.
The function may assume that the next character in the file stream
is a digit,
and should consume characters from the stream until a
character that is neither a digit nor a comma is encountered
(which should be returned to the stream for later processing), or end of file.
Your function should write an error message and exit for misplaced commas.
You may not call any other functions, except for
{\tt fgetc}, {\tt ungetc}, {\tt exit}, {\tt fprintf},
and any helper functions you provide.

For fewer points, use function prototype
\begin{verbatim}
    void read_int(FILE* in, char* lexeme);
\end{verbatim}
where your function may assume that character buffer {\tt lexeme} will
be large enough to hold the digits.
For extra points, use function prototype
\begin{verbatim}
    void read_int(FILE* in, char* lexeme, unsigned max_length);
\end{verbatim}
where {\tt max\_length} (which will be at least 4)
is the size of the character buffer.
Give an error message and exit if the digits will not fit into the buffer.

Below is a table of examples,
showing the file contents (from the current position onward)
immediately before calling \verb|read_int|,
and the contents of the buffer {\tt lexeme} on return.
\begin{center}
\begin{tabular}{l|l}
  File & {\tt lexeme} (or error message)
\\ \hline
  \tt 42 43 44& \tt 42
\\
  \tt 12,345,678+3 & \tt 12345678
\\
  \tt 123,456,789,012,345zzzzz & \tt 12345789012345
\\
  \tt 00,004e7 & \tt 00004
\\[2mm]
  \tt 12345 & \tt Error: expected comma
\\
  \tt 12,13,14 & \tt Error: expected digit
\\
\end{tabular}
\end{center}

%============================================================
\section*{Question~2~\hfill 15 points}
%============================================================

Write a regular expression,
or draw a nondeterministic finite automaton,
that accepts an HTML open tag of the form
\verb|<tagname>|
where a \verb|tagname| consists
of one or more lower-case letters,
followed by an optional digit from 1 to 9.
The following are examples of words that should be accepted:
\begin{verbatim}
    <a>
    <bogusname3>
    <h1>
    <table>
    <ul>
\end{verbatim}
Note that you do not need to check that this is a valid HTML tag.

\end{document}
