\documentclass[10pt]{article}
\usepackage{fullpage}
\usepackage{url}
\usepackage{color}
\usepackage{listings}
\usepackage{framed}
\usepackage{booktabs}

\definecolor{dkgreen}{rgb}{0,0.6,0}
\definecolor{dkred}{rgb}{0.6,0,0}
\definecolor{dkblue}{rgb}{0,0,0.7}

\usepackage{tikz}
\usetikzlibrary{automata, positioning, arrows}
\tikzset{%
  node distance=2.5cm,
  initial text={},
  every state/.style={
    semithick},
  double distance=2pt,  % Accept state appearance
  every edge/.style={
    draw,
    ->,
    >=stealth',
    auto,
    semithick} %
}

\lstdefinestyle{jvm}{
  % aboveskip=3mm,
  % belowskip=3mm,
  xleftmargin=2em,
  % xrightmargin=2em,
  showstringspaces=false,
  columns=flexible,
  basicstyle={\ttfamily},
  numbers=left,
  moredelim=[s][\color{black}]{Ljava}{;},
  morecomment=[l][\color{dkgreen}]{;},
  morecomment=[l][\color{magenta}]{;;},
  keywords={class,public,static,super,method,code,end},
  keywordstyle=\color{dkblue},
  breaklines=true,
  breakatwhitespace=true,
  tabsize=8
}
% ==========================
\definecolor{dkgreen}{rgb}{0,0.6,0}
\definecolor{dkred}{rgb}{0.6,0,0}
\definecolor{dkblue}{rgb}{0,0,0.7}

\usepackage{tikz}
\usetikzlibrary{automata, positioning, arrows}
\tikzset{%
  node distance=2.5cm,
  initial text={},
  every state/.style={
    semithick},
  double distance=2pt,  % Accept state appearance
  every edge/.style={
    draw,
    ->,
    >=stealth',
    auto,
    semithick} %
}

\lstdefinestyle{jvm}{
  % aboveskip=3mm,
  % belowskip=3mm,
  xleftmargin=2em,
  % xrightmargin=2em,
  showstringspaces=false,
  columns=flexible,
  basicstyle={\ttfamily},
  numbers=left,
  moredelim=[s][\color{black}]{Ljava}{;},
  morecomment=[l][\color{dkgreen}]{;},
  morecomment=[l][\color{magenta}]{;;},
  keywords={class,public,static,super,method,code,end},
  keywordstyle=\color{dkblue},
  breaklines=true,
  breakatwhitespace=true,
  tabsize=8
}

\title{COM S 440/540 Homework 4}
\date{}
\author{LL Parsing}

\begin{document}

\maketitle

\noindent
Reminder: present your own work and properly cite any sources used.
Solutions should be written satisfying the \emph{other student viewpoint},
and must be prepared using \LaTeX.
\renewcommand{\thepage}{~}
%============================================================
\section*{Question~1~\hfill 30 points}
%============================================================

Consider the grammar below,
for (simple) assignments in C, with pointer dereferencing
and (simple) pointer arithmetic.
Symbols \emph{literal}, \emph{ident}, \verb|=|, \verb|+|, \verb|-|, \verb|*|,
\verb|(|, \verb|)|,
and \verb|;|
are terminals.
\begin{eqnarray*}
  S & \rightarrow & L ~ \verb|=| ~ \mathit{literal} ~ \verb|;|
\\
  L & \rightarrow & \mathit{ident}
\\
	L & \rightarrow & \verb|*| ~ P
\\
	P & \rightarrow & L
\\
  P & \rightarrow & \verb|(| ~ A ~ \verb|)|
\\
  A & \rightarrow & L ~ O ~ \mathit{literal}
\\
  O & \rightarrow & \verb|+|
\\
  O & \rightarrow & \verb|-|
\end{eqnarray*}
Build the LL(1) parse table $M$ for this grammar.
For full credit, show how each entry was obtained.
%============================================================
% \begin{framed}
\begin{table}[h]
    \centering
    \begin{tabular}{c|c|c|c|c|c}
        \toprule
            & $ident$ & $+$ & $-$ & $*$ & (   \\
        \midrule
        $S$ & (1)     &     &     & (1) &     \\
        $L$ & (2)     &     &     & (3) &     \\
        $P$ & (4)     &     &     & (4) & (5) \\
        $A$ & (6)     &     &     & (6) &     \\
        $O$ &         & (7) & (8) &     &     \\
        \bottomrule
    \end{tabular}
\end{table}
% \end{framed}
%============================================================

%============================================================
\section*{Question~2~\hfill 20 points}
%============================================================

Based on your answer to the previous question, 
write a predictive, recursive-descent parser,
as pseudocode for each non-terminal grammar symbol.

%============================================================
\begin{framed}

\begin{lstlisting}{style=jvm}
    S() {
    if( next == ident || next == '*') {
        L();
        match(=);
        match(literal);
        match(;)
        return;
       }
       sytaxError();
    }

    L() {
        if( next == ident) {
            match(ident);
            return;
        }
        else if(next == '*') {
            match('*');
            P();
        }
        sytaxError();
    }

    P() {
        if (next == ident || next == '*') {
            L();
            return;
        }
        else if(next == '(' ){
            match('(');
            A();
            match(')');
            return;
        }
        sytaxError();
    }

    A() {
        if (next == ident || next == '*') {
            L();
            O();
            match(literal);
            return;
        }
        sytaxError();
    }

    O() {
        if (next == '+') {
            match('+');
            return;
        }
        else if (next == '-') {
            match('-');
            return;
        }
        sytaxError();
    }
\end{lstlisting}

\end{framed}
%============================================================

%============================================================
\section*{Question~3~~(Extra credit for students in 440)\hfill 10 points}
%============================================================

Give a modified version of the grammar
(you may introduce new grammar symbols if needed)
that remains in LL(1),
to allow for arbitrary depth struct member selection.
Specifically,
in place of an \emph{ident},
your modified grammar should allow
an \emph{ident} followed by zero or more
occurrences of sequence \verb|.|\emph{ident}.
For example, the new grammar should accept
the following sentences.
\begin{verbatim}
  i = 3;
  point.x = 3;
  *points.xarray = 3;
  *(points.xarray+7) = 3;
  *(*(strange.twod.array.pointer+4)+5) = 3;
\end{verbatim}
Demonstrate that the modified grammar is still in LL(1).
%============================================================
\begin{framed}
Modify the grammar (changes in red):
\begin{eqnarray*}
    S & \rightarrow & L ~ \verb|=| ~ \mathit{literal} ~ \verb|;| \\
    L & \rightarrow & \color{red}{\mathit{ident} ~ I^{\prime}}
    \\
    L & \rightarrow & \verb|*| ~ P
    \\
    P & \rightarrow & L
    \\
    P & \rightarrow & \verb|(| ~ A ~ \verb|)|
    \\
    A & \rightarrow & L ~ O ~ \mathit{literal}
    \\
    O & \rightarrow & \verb|+|
    \\
    O & \rightarrow & \verb|-|
\end{eqnarray*}

And give the grammar of $I^{\prime}$

\begin{eqnarray*}
    I^{\prime} & \rightarrow & \verb|.| ~ \mathit{ident} ~ I^{\prime}
    \\
    I^{\prime} & \rightarrow & \epsilon
\end{eqnarray*}

It only adds two new entries  $M(I^\prime, \verb|.|) = (9)$ and $M(I^\prime, \$) = (10)$ to the parse table, so it is still LL(1).
\end{framed}
%============================================================

\end{document}
