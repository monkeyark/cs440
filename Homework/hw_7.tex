\documentclass[10pt]{article}
\usepackage{fullpage}
\usepackage{url}
\usepackage{color}
\usepackage{listings}
\usepackage{framed}
\usepackage{amsmath}

\definecolor{dkgreen}{rgb}{0,0.6,0}
\definecolor{dkred}{rgb}{0.6,0,0}
\definecolor{dkblue}{rgb}{0,0,0.7}

\usepackage{tikz}
\usetikzlibrary{automata, positioning, arrows}
\tikzset{%
  node distance=2.5cm,
  initial text={},
  every state/.style={
    semithick},
  double distance=2pt,  % Accept state appearance
  every edge/.style={
    draw,
    ->,
    >=stealth',
    auto,
    semithick} %
}

\lstdefinestyle{jvm}{
  % aboveskip=3mm,
  % belowskip=3mm,
  xleftmargin=2em,
  % xrightmargin=2em,
  showstringspaces=false,
  columns=flexible,
  basicstyle={\ttfamily},
  numbers=left,
  moredelim=[s][\color{black}]{Ljava}{;},
  morecomment=[l][\color{dkgreen}]{;},
  morecomment=[l][\color{magenta}]{;;},
  keywords={class,public,static,super,method,code,end},
  keywordstyle=\color{dkblue},
  breaklines=true,
  breakatwhitespace=true,
  tabsize=8
}

\title{COM S 440/540 Homework 7}
\date{}
\author{Code generation}

\begin{document}

\maketitle

\noindent
Reminder: present your own work and properly cite any sources used.
Solutions should be written satisfying the \emph{other student viewpoint},
and must be prepared using \LaTeX.
\renewcommand{\thepage}{~}
%============================================================
\section*{Question~1~\hfill 20--30 points}
%============================================================

Suppose we define a generalized ``post-increment'' operator, \verb|#|,
that increments a variable by an expression \emph{after} returning
the value of the variable.
More formally, we allow expressions of the form
\[
  E ~\rightarrow~~\mathit{ident} ~ \verb|#| ~ E_1
\]
which evaluates expression $E_1$ exactly once,
returns the value of the identifier,
and then adds the value of $E_1$ to the identifier.
Give the semantic rules for this production,
to (incrementally) translate into three-address code.
For additional points, also give the semantic rules
to (incrementally) translate into stack machine code.

%============================================================
\begin{framed}
The semantic rules for the grammar to translate to three-address code are as follows:

\begin{lstlisting}[style=jvm]
  E.addr = new_temporary();
  output_instruction(E.addr, '=', E1.addr);
  E2.addr = symbol_table_entry_for(ident.name)
  E3.addr = new_temporary();
  output_instruction(E3.addr, '=', E1.addr, '+', E2.addr);
  output_instruction(E2.addr '=', E3.addr);
\end{lstlisting}

The semantic rules for the grammar to translate to stack machine code are as follows:

\begin{lstlisting}[style=jvm]
  output_instruction("push ", ident.name);
  output_instruction("duplicate");
  output_instruction("push ", E1.addr);
  output_instruction("add");
  output_instruction("pop", ident.name);
\end{lstlisting}

\end{framed}
%============================================================

%============================================================
\section*{Question~2~\hfill 20--25 points}
%============================================================

Suppose we are designing a language with short-circuiting
boolean expressions,
and wish to include an \emph{implication} operator, \verb|=>|,
with the following truth table:
\[
  \begin{array}{c|c|c}
    x & y & x ~\verb|=>|~ y
  \\ \hline
    \mathit{false} & \mathit{false} & \mathit{true}
  \\
    \mathit{false} & \mathit{true} & \mathit{true}
  \\
    \mathit{true} & \mathit{false} & \mathit{false}
  \\
    \mathit{true} & \mathit{true} & \mathit{true}
  \end{array}
\]
Give semantic rules
for the production
\[
  B ~\rightarrow~ B_1 ~~\verb|=>|~~ B_2
\]
with short-circuiting.
For additional points,
give the semantic rules using the special fall-through label,
to avoid redundant gotos.
%============================================================
\begin{framed}
The semantic rules for the production $B \rightarrow B_1 \implies B_2$ (i.e. $B \rightarrow \neg B_1 \lor B_2$) are as follows:

\begin{lstlisting}[style=jvm]
B.generate(label t, label f){
  label b1f = (t != fall_through) ? t : new label;
  B1.generate(fall_through, b1f);
  B2.generate(t, f);
  if (f == fall_through)
      output_label(b1f);
}
  
\end{lstlisting}

\end{framed}
%============================================================

%============================================================
\section*{Question~3~\hfill 20--30 points}
%============================================================
Suppose a \emph{repeat-until} loop
has the form
\[
  S \quad \rightarrow \quad \mathsf{repeat} ~~ S_1 ~~\mathsf{until} ~~ B
\]
where $B$ is a boolean expression that short circuits,
and $S_1$ is a non-empty list of statements.
The loop will execute statement(s) $S_1$ one or more times,
stopping when condition $B$ is true.
Give the semantic rules to generate three-address code for a repeat-until loop.
For additional points, also give the semantic rules using backpatching
(you may modify the above production to include markers if necessary).
%============================================================
\begin{framed}
The three-address code rules for the production $S \rightarrow \texttt{repeat } S_1 \texttt{ until } B$ are as follows:


\begin{lstlisting}[style=jvm]
        S.generate(next)
        {
            label loop = new label();
            lable tpart = new label();
            label loop_end = new label();
            output_label(loop);
            S1.generate(next);
            output_label(loop_end);
            B1.generate(tpart, next);
            output_label(tpart);
            output_instruction("goto ", loop);
        }
\end{lstlisting}
\end{framed}

\begin{framed}
Backpatching:

Rewrite the production: $ S \rightarrow \texttt{repeat } M_1 \ S_1 \texttt{ until } M_2 \  B $

\begin{lstlisting}[style=jvm]
            {
            backpatching(S1.nextlist, M2.instr);
            backpatching(B.falselist, M1.instr);
            S.nextlist = B.truelist;
            output_instruction("goto ", M1.instr);
            }
\end{lstlisting}
\end{framed}
%============================================================

\end{document}
