\documentclass[10pt]{article}
\usepackage{fullpage}
\usepackage{url}
\usepackage{color}
\usepackage{listings}
\usepackage{framed}
\usepackage{booktabs}
\usepackage{amsmath}

\definecolor{dkgreen}{rgb}{0,0.6,0}
\definecolor{dkred}{rgb}{0.6,0,0}
\definecolor{dkblue}{rgb}{0,0,0.7}

\usepackage{tikz}
\usetikzlibrary{automata, positioning, arrows}
\tikzset{%
  node distance=2.5cm,
  initial text={},
  every state/.style={
    semithick},
  double distance=2pt,  % Accept state appearance
  every edge/.style={
    draw,
    ->,
    >=stealth',
    auto,
    semithick} %
}

\lstdefinestyle{jvm}{
  % aboveskip=3mm,
  % belowskip=3mm,
  xleftmargin=2em,
  % xrightmargin=2em,
  showstringspaces=false,
  columns=flexible,
  basicstyle={\ttfamily},
  numbers=left,
  moredelim=[s][\color{black}]{Ljava}{;},
  morecomment=[l][\color{dkgreen}]{;},
  morecomment=[l][\color{magenta}]{;;},
  keywords={class,public,static,super,method,code,end},
  keywordstyle=\color{dkblue},
  breaklines=true,
  breakatwhitespace=true,
  tabsize=8
}

\title{COM S 440/540 Homework 5}
\date{}
\author{LR Parsing}

\begin{document}

\maketitle

\noindent
Reminder: present your own work and properly cite any sources used.
Solutions should be written satisfying the \emph{other student viewpoint},
and must be prepared using \LaTeX.
\renewcommand{\thepage}{~}
 
 
For all questions, use the following grammar, where
%============================================================
\emph{lbrace}, \emph{rbrace}, \emph{comma}, and \emph{number}
are terminal symbols (i.e., tokens).

\begin{eqnarray}
  S & \rightarrow & \mathit{lbrace} ~L~ \mathit{rbrace}
\\
  L & \rightarrow & L ~\mathit{comma}~ I
\\
  L & \rightarrow & I
\\
  I & \rightarrow & \mathit{number}
\\
  I & \rightarrow & S
\end{eqnarray}


%============================================================
\section*{Question~1~\hfill 15 points}
%============================================================

Determine the LR(0) \textsc{Closure} of item ~~
$S$ $\rightarrow$ $\mathit{lbrace}$ $\bullet$ $L$ $\mathit{rbrace}$
%============================================================
\begin{framed}
\begin{align*}
\textsc{Closure}(S &\rightarrow \mathit{lbrace} \bullet \  L \  \mathit{rbrace}) = \\
\{ \\
  & S \rightarrow \mathit{lbrace} \bullet \  L \  \mathit{rbrace}, \\
  & L \rightarrow \bullet \  L \ \mathit{comma} \ I, \\
  & L \rightarrow \bullet \  I, \\
  & I \rightarrow \bullet \ \mathit{number}, \\
  & I \rightarrow \bullet \ S, \\
  & S \rightarrow \bullet \ \mathit{lbrace} \  L \  \mathit{rbrace} \\
\}
\end{align*}
\end{framed}
%============================================================

%============================================================
\section*{Question~2~\hfill 10 points}
%============================================================

Using {\tt bison -g}, generate a DOT file of the automaton
for this grammar.
Then, use {\tt dot -Tpdf} to produce a PDF of the automaton.
Either include this in your \LaTeX{} submission, or submit it separately
in Canvas.

%============================================================
\begin{framed}
% \includegraphics[width=\textwidth]{tuple.pdf}
\end{framed}
%============================================================
\newpage
%============================================================
\section*{Question~3~\hfill 40 points}
%============================================================

Using the automaton states as numbered by {\tt bison},
give the LR(0) \textsc{Action} and \textsc{Goto} tables
for this grammar.
Indicate if there are any conflicts.

%============================================================
\begin{framed}
  \centering
  \begin{tabular}{c|ccccc|ccc}
    State & lbrace & rbrace & comma  & number & \$    & S      & L      & I  \\
    \midrule
    0     & s1     &        &        &        &       & 2      &        &    \\
    1     & s1     &        &        & s3     &       & 4      & 5      & 6  \\
    2     &        &        &        &        & acc   &        &        &    \\
    3     & r4     & r4     & r4     & r4     & r4    &        &        &    \\
    4     & r5     & r5     & r5     & r5     & r5    &        &        &    \\
    5     &        & s8     & s9     &        &       &        &        &    \\
    6     & r3     & r3     & r3     & r3     & r3    &        &        &    \\
    8     & r1     & r1     & r1     & r1     & r1    &        &        &    \\
    9     & s1     &        &        & s3     &       & 4      &        & 10 \\
    10    & r2     & r2     & r2     & r2     & r2    &        &        &    \\
  \end{tabular}
\end{framed}
%============================================================

%============================================================
\section*{Question~4~~(extra credit for students in 440)\hfill 15 points}
%============================================================

Trace the execution of your LR(0) parser on the input string
\begin{center}
\emph{lbrace}  ~\emph{lbrace}  ~\emph{number}  ~\emph{rbrace}
 ~\emph{comma}  ~\emph{number} ~\$
\end{center}
by indicating the stack contents, remaining input, and
action(s) taken at each step.
%============================================================
\begin{framed}
  \begin{tabular}{l|l|l|l}
    Stack                             & Input                                                         & Action & Reduction                                              \\
    \midrule
    0                                 & \emph{lbrace}  ~\emph{lbrace}  ~\emph{number}  ~\emph{rbrace}
    ~\emph{comma}  ~\emph{number} ~\$ & s1                                                            &                                                                 \\
    0 1                               & \emph{lbrace}  ~\emph{number}  ~\emph{rbrace}
    ~\emph{comma}  ~\emph{number} ~\$ & s1                                                            &                                                                 \\
    0 1 1                             & \emph{number}  ~\emph{rbrace}
    ~\emph{comma}  ~\emph{number} ~\$ & s3                                                            &                                                                 \\
    0 1 1 3                           & \emph{rbrace} ~\emph{comma}  ~\emph{number} ~\$               & r4     & $I \rightarrow \mathit{number}$                        \\
    0 1 1 6                           & \emph{rbrace} ~\emph{comma}  ~\emph{number} ~\$               & r3     & $L \rightarrow I$                                      \\
    0 1 1 5                           & \emph{rbrace} ~\emph{comma}  ~\emph{number} ~\$               & s8     &                                                        \\
    0 1 1 5 8                         & \emph{comma}  ~\emph{number} ~\$                              & r1     & $S \rightarrow \mathit{lbrace} \  L \ \mathit{rbrace}$ \\
    0 1 4                             & \emph{comma}  ~\emph{number} ~\$                              & r5     & $I \rightarrow S$                                      \\
    0 1 6                             & \emph{comma}  ~\emph{number} ~\$                              & r3     & $L \rightarrow I$                                      \\
    0 1 5                             & \emph{comma}  ~\emph{number} ~\$                              & s9     &                                                        \\
    0 1 5 9                           & \emph{number} ~\$                                             & s3     &                                                        \\
    0 1 5 9 3                         & ~\$                                                           & r4     & $I \rightarrow \mathit{number}$                        \\
    0 1 5 9 10                        & ~\$                                                           & r2     & $L \rightarrow L \  \mathit{comma} \  I$               \\
    0 1 5                             & ~\$                                                           & Reject &                                                        \\
  \end{tabular}
\end{framed}

%============================================================

\end{document}
