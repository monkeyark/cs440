\documentclass[10pt]{article}
\usepackage{fullpage}
\usepackage{url}
\usepackage{color}
\usepackage{listings}
\usepackage{framed}

\definecolor{dkgreen}{rgb}{0,0.6,0}
\definecolor{dkred}{rgb}{0.6,0,0}
\definecolor{dkblue}{rgb}{0,0,0.7}

\usepackage{tikz}
\usetikzlibrary{automata, positioning, arrows}
\tikzset{%
  node distance=2.5cm,
  initial text={},
  every state/.style={
    semithick},
  double distance=2pt,  % Accept state appearance
  every edge/.style={
    draw,
    ->,
    >=stealth',
    auto,
    semithick} %
}

\lstdefinestyle{jvm}{
  % aboveskip=3mm,
  % belowskip=3mm,
  xleftmargin=2em,
  % xrightmargin=2em,
  showstringspaces=false,
  columns=flexible,
  basicstyle={\ttfamily},
  numbers=left,
  moredelim=[s][\color{black}]{Ljava}{;},
  morecomment=[l][\color{dkgreen}]{;},
  morecomment=[l][\color{magenta}]{;;},
  keywords={class,public,static,super,method,code,end},
  keywordstyle=\color{dkblue},
  breaklines=true,
  breakatwhitespace=true,
  tabsize=8
}

\title{COM S 440/540 Homework 3}
\date{}
\author{Grammars}

\begin{document}

\maketitle

\noindent
Reminder: present your own work and properly cite any sources used.
Solutions should be written satisfying the \emph{other student viewpoint},
and must be prepared using \LaTeX.
\renewcommand{\thepage}{~}
 
\noindent 
{\bf For questions 1 and 2}, use the following grammar, where
%============================================================
\emph{lbrace}, \emph{rbrace}, \emph{comma}, and \emph{number}
are terminal symbols (i.e., tokens).

\begin{eqnarray}
  S & \rightarrow & \mathit{lbrace} ~L~ \mathit{rbrace}
\\
  L & \rightarrow & L ~\mathit{comma}~ I
\\
  L & \rightarrow & I
\\
  I & \rightarrow & \mathit{number}
\\
  I & \rightarrow & S
\end{eqnarray}


%============================================================
\section*{Question~1~\hfill 15 points}
%============================================================

Determine
  $\textsc{First}(S)$,
  $\textsc{First}(L)$,
  $\textsc{First}(I)$,
  $\textsc{Follow}(S)$,
  $\textsc{Follow}(L)$, and
  $\textsc{Follow}(I)$.
%============================================================
\begin{framed}
\begin{eqnarray*}
    \textsc{First}(S) & = & \{lbrace\} \\
    \textsc{First}(L) & = & \{lbrace, number \} \\
    \textsc{First}(I) & = & \{lbrace, number \} \\
    \textsc{Follow}(S) & = & \{comma, \$, rbrace\} \\
    \textsc{Follow}(L) & = & \{rbrace, comma\} \\
    \textsc{Follow}(I) & = & \{rbrace, comma\}
\end{eqnarray*}
\end{framed}
%============================================================

%============================================================
\section*{Question~2~\hfill 15 points}
%============================================================

Write an equivalent grammar (with the same precedence
of operators) that does not use left recursion.
%============================================================
\begin{framed}
\begin{eqnarray*}
    S & \rightarrow & \mathit{lbrace} ~L~ \mathit{rbrace}
    \\
    L & \rightarrow & I ~\mathit{comma}~ L
    \\
    L & \rightarrow & I
    \\
    I & \rightarrow & \mathit{number}
    \\
    I & \rightarrow & S
\end{eqnarray*}
\end{framed}
%============================================================
\section*{Question~3~\hfill 20 points}
%============================================================

Write an unambiguous grammar to describe the language of 
``variables'' in C.
For this question, a ``variable'' is an identifier
that could be an array in several dimensions,
where the array indexes are also variables.
The following are examples of legal sentences.
\begin{quote}
  \emph{ident} \\
  \emph{ident}[\emph{ident}] \\
  \emph{ident}[ \emph{ident}[\emph{ident}][\emph{ident}] ]
  [ \emph{ident} ]
  [ \emph{ident}[\emph{ident}[\emph{ident}]][\emph{ident}] ]
  [ \emph{ident}[\emph{ident}][\emph{ident}][\emph{ident}] ]
\end{quote}
Your grammar should use $\mathit{ident}$, $[$, and $]$ as
terminal symbols.
%============================================================
\begin{framed}
\begin{eqnarray*}
    A & \rightarrow & A[A] \\
    A & \rightarrow & ident
\end{eqnarray*}
\end{framed}
%============================================================

\end{document}
