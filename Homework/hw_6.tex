\documentclass[10pt]{article}
\usepackage{fullpage}
\usepackage{url}
\usepackage{color}
\usepackage{listings}
\usepackage{framed}

\definecolor{dkgreen}{rgb}{0,0.6,0}
\definecolor{dkred}{rgb}{0.6,0,0}
\definecolor{dkblue}{rgb}{0,0,0.7}

\usepackage{tikz}
\usetikzlibrary{automata, positioning, arrows}
\tikzset{%
  node distance=2.5cm,
  initial text={},
  every state/.style={
    semithick},
  double distance=2pt,  % Accept state appearance
  every edge/.style={
    draw,
    ->,
    >=stealth',
    auto,
    semithick} %
}

\lstdefinestyle{jvm}{
  % aboveskip=3mm,
  % belowskip=3mm,
  xleftmargin=2em,
  % xrightmargin=2em,
  showstringspaces=false,
  columns=flexible,
  basicstyle={\ttfamily},
  numbers=left,
  moredelim=[s][\color{black}]{Ljava}{;},
  morecomment=[l][\color{dkgreen}]{;},
  morecomment=[l][\color{magenta}]{;;},
  keywords={class,public,static,super,method,code,end},
  keywordstyle=\color{dkblue},
  breaklines=true,
  breakatwhitespace=true,
  tabsize=8
}

\title{COM S 440/540 Homework 6}
\date{}
\author{Attribute grammars}

\begin{document}

\maketitle

\noindent
Reminder: present your own work and properly cite any sources used.
Solutions should be written satisfying the \emph{other student viewpoint},
and must be prepared using \LaTeX.
\renewcommand{\thepage}{~}
 
 
For all questions, use the following grammar, where
%============================================================
\emph{lbrace}, \emph{rbrace}, \emph{comma}, and \emph{number}
are terminal symbols (i.e., tokens).
\begin{eqnarray*}
  S & \rightarrow & \mathit{lbrace} ~L~ \mathit{rbrace}
\\
  L & \rightarrow & L ~\mathit{comma}~ I
\\
  L & \rightarrow & I
\\
  I & \rightarrow & \mathit{number}
\\
  I & \rightarrow & S
\end{eqnarray*}


%============================================================
\section*{Question~1~\hfill 20 points}
%============================================================

For the grammar above, give semantic rules to build
a synthesized attribute named \emph{max},
which keeps track of the largest \emph{number} appearing anywhere
within the set.
You may introduce other synthesized attributes,
and give rules for them,
as needed.
For example, the sentence
\begin{verbatim}
  { 3, {{1}, 4}, {1, 5}, {{{9}, 2}, 6}, 5 }
\end{verbatim}
should have a \emph{max} attribute equal to 9.
(Assume lexemes ``\verb|{|'', ``\verb|}|'', and ``\verb|,|''
correspond to tokens \emph{lbrace}, \emph{rbrace},
and \emph{comma}, respectively).
%============================================================
\begin{framed}
\centering
\begin{tabular}{ll}
    Production                                              & Rule \\
    \hline
    $S  \rightarrow  \mathit{lbrace} ~L~ \mathit{rbrace}$   & $S.\mathit{max} = L.\mathit{max}$ \\
    $L  \rightarrow  L^\prime ~\mathit{comma}~ I$           & $L.\mathit{max} = \max(L^\prime.\mathit{max}, I.\mathit{max})$ \\
    $L  \rightarrow  I$                                     & $L.\mathit{max} = I.\mathit{max}$ \\
    $I  \rightarrow  \mathit{number}$                       & $I.\mathit{max} = \mathit{number}.\mathit{value}$ \\
    $I  \rightarrow  S$                                     & $I.\mathit{max} = S.\mathit{max}$ \\
\end{tabular}
\end{framed}
%============================================================
\section*{Question~2~\hfill 30 points}
%============================================================
For the grammar above, give semantic rules to build
an inherited attribute named \emph{depth},
which keeps track of the depth of each item:
each item in a set should have depth one greater
than the depth of the set containing it.
For example, for the input sentence
\begin{verbatim}
  { 1, {{2}, 3}, {4, 5}, {{{6}, 7}, 8}, 9 }
\end{verbatim}
we should obtain the following depths.
\begin{center}
\begin{tabular}{lrcrlrc}
  Item & & Depth & ~ ~ ~ & Item & & Depth \\
  \cline{1-1} \cline{3-3} \cline{5-5} \cline{7-7}
  \verb|{ 1, {{2}, 3}, {4, 5}, {{{6}, 7}, 8}, 9 }| && 0
  && \verb|4| && 2 \\
  \verb|1|  && 1
  && \verb|5| && 2 \\
  \verb|{{2}, 3}| && 1
  && \verb|{{6}, 7}| && 2 \\
  \verb|{4, 5}| && 1
  && \verb|8| && 2 \\
  \verb|{{{6}, 7}, 8}| && 1
  && \verb|2| && 3 \\
  \verb|9| && 1
  && \verb|{6}| && 3 \\
  \verb|{2}| && 2
  && \verb|7| && 3 \\
  \verb|3| && 2
  && \verb|6| && 4
\end{tabular}
\end{center}

%============================================================
% We should the the table:
\begin{framed}
\centering
\begin{tabular}{l|l}
    Production                                                & Rule \\
    \hline
    $S  \rightarrow  \mathit{lbrace} ~L~ \mathit{rbrace}$     & $L.depth = S.depth + 1$ \\
    \hline
    $L  \rightarrow  L^\prime ~\mathit{comma}~ I$             & $L^\prime.depth = L.depth$ \\
                                                              & $I.depth = L.depth$ \\
    \hline
    $L  \rightarrow  I$                                       & $I.depth = L.depth$ \\
    \hline
    $I  \rightarrow  \mathit{number}$                         & $\{number.depth = I.depth\}$ \\
    \hline
    $I  \rightarrow  S$                                       & $S.depth = I.depth$ \\
\end{tabular}
\end{framed}
%============================================================

\end{document}
